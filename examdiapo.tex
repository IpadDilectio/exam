\documentclass[11pt,xcolor=dvipsname,ignorenonframetext,handout]{beamer}
%\documentclass{beamer}
%
% Choose how your presentation looks.
%
% For more themes, color themes and font themes, see:
% http://deic.uab.es/~iblanes/beamer_gallery/index_by_theme.html
%
\mode<presentation>
{
  \usetheme{Boadilla}      % or try Darmstadt, Madrid, Warsaw, ...
  %\usetheme{Szeged}      % or try Darmstadt, Madrid, Warsaw, ...
  %\usetheme{CambridgeUS}      % or try Darmstadt, Madrid, Warsaw, ...
  \usecolortheme{rose} % or try albatross, beaver, crane, ...
  %\usecolortheme{seahorse} % or try albatross, beaver, crane, ...
  \usefonttheme{default}  % or try serif, structurebold, ...
  %\usefonttheme{serif}
  \setbeamertemplate{navigation symbols}{}
  \setbeamertemplate{caption}[numbered]
}


\usepackage[utf8]{inputenc}
\usepackage[T1]{fontenc}
\usepackage[french]{babel}
\usepackage{enumerate}
\frenchbsetup{StandardItemizeEnv=true}
\frenchbsetup{StandardEnumerateEnv=true}
\usepackage{graphicx}
\usepackage{hyperref}
\hypersetup{
     colorlinks   = true,
     citecolor    = gray
}
\hypersetup{linkcolor=blue}
\title{L'examen de conscience et l'oraison du matin}
\subtitle{Comment augmenter notre vertu de douceur à notre égard ?}
\author{Dilectio}
\institute{}
\date{\today}
\AtBeginSection[]{
    \begin{frame}{Sommaire}
        \small
%        \tableofcontents[currentsection, hideothersubsections]
%    \end{frame}
\begin{columns}[t]
    \begin{column}{5cm}
        \tableofcontents[sections={1-4},currentsection, hideothersubsections]
    \end{column}
    \begin{column}{5cm}
        \tableofcontents[sections={5-8},currentsection,hideothersubsections]
    \end{column}
\end{columns}
\end{frame}
}
% La commande suivante permet de mettre le texte en indice supérieur
\newcommand{\supe}[1]{\textsuperscript{#1}}
\begin{document}

\begin{frame}
  \titlepage
\end{frame}

% Uncomment these lines for an automatically generated outline.
%\begin{frame}{Table des matières}
%  \tableofcontents
%\end{frame}
\section{Introduction}
\begin{frame}{Introduction~: comment mieux gérer notre temps~?}
    \begin{itemize}
        \rightskip=0pt\leftskip=0pt
        \item Nous vivons dans une société où il est facile d'être débordé par l'ensemble de nos activités~;
        \item Il existe de nombreuses méthodes de gestion du temps, telle par exemple celle de David Hallen, la méthode GTD, Getting Things Done.
        \item Il est dans l'Esprit du Temps, le Zeitgeist, de nous faire croire que ce serait l'amélioration des techniques qui améliorerait notre vie de tous les jours, dont les techniques liés à l'informatique.
        \item Je les crois assez peu efficaces même si je ne dis pas qu'il faut tout rejeter. Je crois qu'il ne faut pas confondre l'utile de l'essentiel.
        \item Je préfère suivre les conseils de Saint François de Sales~: il vaut mieux \textbf{changer notre cœur} que changer nos techniques.
    \end{itemize}  
\begin{frame}{Introduction~: comment mieux gérer notre temps~?}
    \begin{itemize}
        \rightskip=0pt\leftskip=0pt
        \item Au risque de vous surprendre un peu, je vais essayer de vous donner envie de mieux connaître ce que dit Saint François de Sales sur l'\textbf{examen de conscience}.
        \item Je crois que nous avons là, ce qu'il faut pour réussir à changer notre cœur pour mieux gérer notre temps.
        \item Je vous inciterais à regarder aussi ce qu'il dit sur l'\textbf{oraison du matin}.
        \item Comme ces 2 exercices demandent de développer 3 vertus, je vous les présenterai rapidement.
        \item Enfin, je vous indiquerai les entraves qui peuvent nous empêcher de bien conduire ces 2 exercices.
    \end{itemize}
\end{frame}
\begin{frame}{Introduction~: comment mieux gérer notre temps~?}
    \begin{itemize}
        \rightskip=0pt\leftskip=0pt
        \item Comme j'ai très peu de temps pour vous présenter un sujet assez riche, je vais essayer de survoler le thème en suscitant votre désir de l'approfondir.
        \item En même temps, comme il n'est pas correct de susciter un désir et de le frustrer aussitôt, je vous mets à disposition sur internet, un article plus développé.
        \item Vous le trouverez sur mon site : \href{http://www.dilectio.fr}{{\color{blue}dilectio.fr}}, à l'article \textbf{Examen de conscience}.
        \item Vous y trouverez aussi de nombreux autres articles qui peuvent aider à mieux comprendre ce thème.
\end{frame}
\section{L'examen de conscience}
\begin{frame}{L'examen de conscience}
\rightskip=0pt\leftskip=0pt
Dans son livre, \emph{Introduction à la vie dévote}, Saint François de Salle nous propose plusieurs étapes pour réaliser notre examen de conscience chaque soir. Je suis attentivement son texte même si je le modifie pour le rendre plus lisible aujourd'hui. Voici les étapes
    \begin{itemize}
        \rightskip=0pt\leftskip=0pt
        \item 
\end{frame}
\begin{frame}{Rappel}
    \rightskip=0pt\leftskip=0pt
    Dans le cours sur la liberté, nous avons vu \textbf{les degrés de liberté}~:
    \begin{itemize}
        \rightskip=0pt\leftskip=0pt
        \item \textcolor{blue}{\textbf{Le degré 0}} de liberté consistait dans le libre arbitre où nous pouvions agir ou ne pas agir, faire le bien, faire le mal, ou ne rien faire.
        \item \textcolor{blue}{\textbf{Le plus haut degré}} de liberté était alors avec Thomas d'Aquin, le pouvoir confié à l'homme de faire le bien.
    \end{itemize}
\end{frame}
\begin{frame}{Nouvelle distinction conceptuelle}
    %\rightskip=0pt\leftskip=0pt
    %En nous rappelant cela, nous pouvons maintenant mettre en place la distinction conceptuelle suivante~:
    \begin{itemize}
        \rightskip=0pt\leftskip=0pt
        \item \textcolor{blue}{\textbf{La liberté d'indifférence~:}} c'est le fait qu'à chaque moment nous pouvons poser un choix indépendant de notre passé et qui ne se soucie pas de notre avenir, comme si nous pouvions nous extraire de notre propre durée. On renonce alors à se soucier du Bien Commun. On ne se soucie alors que du bien égoïste immédiat apparent.
        \item \textcolor{blue}{\textbf{La liberté de qualité~:}} c'est le fait que nous choisissons de faire le bien en tenant compte de ce que nous sommes et des conséquences futures sur nous-même et sur les autres. On se soucie alors du Bien Commun.
    \end{itemize}
\end{frame}
\begin{frame}{Liberté d'indifférence vs liberté de qualité}
    {
        \begin{figure}
          \centering
          \includegraphics[width = 0.8\textwidth]{indifference}
        \end{figure}
          }    
\end{frame}
\begin{frame}{Introduction~: le bonheur est le fruit de la liberté de qualité}
    {
        \begin{figure}
          \centering
          \includegraphics[width = 1.0\textwidth]{samfamily}
        \end{figure}
          }    
\end{frame}
\begin{frame}{Introduction~: le bonheur est le fruit de la liberté de qualité}
    {
        \begin{figure}
          \centering
          \includegraphics[width = 1.0\textwidth]{beaute1}
        \end{figure}
          }    
\end{frame}
\begin{frame}{Introduction~: le bonheur est le fruit de la liberté de qualité}
    {
        \begin{figure}
          \centering
          \includegraphics[width = 0.65\textwidth]{beaute2}
        \end{figure}
          }    
\end{frame}
\begin{frame}{Introduction~: problème}
    \begin{itemize}
        \rightskip=0pt\leftskip=0pt
        \item Par les actes que nous posons, nous fortifions ou non nos vertus. En fortifiant nos vertus, nous rendons possibles notre bonheur, en ne le faisant pas nous le mettons en danger.
        \item Il nous arrive de poser des actes vertueux, mais nous constatons aussi qu'il nous arrive de ne pas le faire.
        \item Qu'est-ce qui nous permet de les poser~?
        \item Pourquoi ne les posons-nous pas toujours~?
    \end{itemize}
\end{frame}{Introduction}
\begin{frame}{Introduction}
    \begin{itemize}
        \rightskip=0pt\leftskip=0pt
        \item C'est \textbf{notre conscience} qui nous permet de poser des actes vertueux, car c'est elle qui nous indique ce que nous devons faire.
        \item C'est pourquoi il est important de mieux comprendre maintenant ce qu'est \textbf{la conscience}~;
        \item Et pourquoi nous ne l'écoutons pas toujours.
    \end{itemize}
\end{frame}
\begin{frame}{Remarques importantes}
    \begin{itemize}
    \rightskip=0pt\leftskip=0pt    
    \item La conception que je vous présente dans ce cours relève de ce que la tradition philosophique appelle l'\textcolor{blue}{\textbf{eudémonisme}}.
    \item Cette conception considère que le bonheur est le but recherché par l'homme. La liberté vise alors, avec toute la marge de manœuvre qui nous est donnée, la réalisation de ce bonheur. 
    \item Cela suppose un développement de l'intelligence théorique et pratique car l'intelligence nous permet de voir les choses telles qu'elles sont, et ainsi de ne pas nous tromper de cible. 
    \item La morale n'est alors pas considérée comme un ensemble d'interdits et d'obligations qui viendraient de l'extérieur, mais plutôt comme \textcolor{blue}{\textbf{la boussole intérieure qui nous permet d'éviter de nous tromper de cible}}.
    \item Cette boussole agit surtout en nous poussant à fortifier nos vertus qui vont réaliser en nous \textbf{l'excellence de notre liberté}~: \textcolor{blue}{\textbf{la liberté de qualité}}.
    \end{itemize}
\end{frame}
\section{La conscience psychologique}
\begin{frame}{La conscience psychologique}
    {
        \begin{figure}
          \centering
          \includegraphics[width = 1\textwidth]{observer}
        \end{figure}
          }    
\end{frame}
\begin{frame}{La conscience psychologique}
    \begin{itemize}
        \rightskip=0pt\leftskip=0pt
        \item On appelle \textbf{conscience psychologique}, \textcolor{blue}{l'acte de connaissance qui porte sur l'objet qui nous est donné}. Cet objet peut être \textbf{un objet extérieur}, cette table, cette chaise, ou \textbf{un objet intérieur}, cette émotion ou ce sentiment.
        \item C'est elle qui permet de savoir qu'un objet est bien présent en face de moi, ou qu'une émotion, un sentiment, ou une pensée, est présente en moi.
        \item On parle du \textcolor{blue}{\textbf{regard de la conscience}}
        \item \textbf{Perdre conscience}, signifie alors ne plus savoir ce qui est présent à mes côtés ou en moi.
        \item \textbf{Prendre conscience}, c'est réaliser grâce à un acte de l'intelligence que quelque chose est présent alors qu'auparavant, \textbf{par défaut d'attention}, nous ne l'avions pas remarquée.
    \end{itemize}
\end{frame}
\begin{frame}{La conscience psychologique}
    \begin{itemize}
        \rightskip=0pt\leftskip=0pt
        \item Cette conscience se divise donc en \textcolor{blue}{conscience des objets extérieurs} et en \textcolor{blue}{conscience intime de soi}.
        \item Elle est d'abord connaissance de l'existence de l'objet, et peut se développer, par l'intelligence et la volonté, en connaissance plus approfondie de l'objet.
        \item C'est donc l'acte qui consiste à observer et à reconnaître ce qui nous entoure ou ce qui se passe en nous, qui peut se transformer en recherche d'une plus grande connaissance.
        \item Nous sommes donc plus ou moins conscient de ce qui nous entoure ou de ce que nous sommes. Il y a des \textbf{degrés de conscience}, comme il y a des degrés de liberté. Plus nous connaissons ce qui nous entoure et ce que nous sommes, plus nous sommes conscients.
    \end{itemize}
\end{frame}
\section{La conscience morale}
\begin{frame}{La conscience morale}
    {
        \begin{figure}
          \centering
          \includegraphics[width = 1\textwidth]{bonte}
        \end{figure}
          }    
\end{frame}
\begin{frame}{Définition de la conscience morale}
    \begin{itemize}
        \rightskip=0pt\leftskip=0pt
        \item Thomas d'Aquin définit la conscience morale comme \textbf{l'application de la science à un acte particulier}, (De Veritate, q17,a1).
        \item Cela vient, dit-il (La Somme q79,a13), de l'étymologie même du mot latin \textbf{conscientia}~: \textbf{cum alio scientia} (connaissance avec un autre).
        \item L'application de la science à un acte particulier, est elle-même \textbf{un acte}. C'est \textbf{un acte d'attention} à la règle que nous donne notre intelligence, puis \textbf{un acte d'application} de ce que nous dit l'intelligence à ce que nous faisons.
        \item On parle de \textcolor{blue}{\textbf{voix de la conscience}}.
        \item C'est un acte de l'intelligence qui consiste à saisir le caractère bon ou mauvais d'une action.
    \end{itemize}
\end{frame}
\begin{frame}{Caractéristiques de l'acte de conscience}
    Thomas d'Aquin précise (Somme q79,a13) que cet acte peut se faire selon \textbf{6 opérations}. La conscience peut~:
    \begin{itemize}
        \rightskip=0pt\leftskip=0pt
        \item \textcolor{blue}{\textbf{Attester}}~: lorsque nous reconnaissons que nous avons accompli ou non telle action~;
        \item \textcolor{blue}{\textbf{Inciter} ou \textbf{obliger}}~: lorsque nous jugeons grâce à elle qu'il faut accomplir ou non une action~;
        \item \textcolor{blue}{\textbf{Excuser}, \textbf{accuser} ou \textbf{reprocher}}~: lorsque nous jugeons grâce à elle que ce qui a été fait était bon ou non.
    \end{itemize}
\end{frame}
\begin{frame}{Caractéristiques de l'acte de conscience}
    L'intelligence par notre conscience intervient donc avant l'action, pendant l'action, et après l'action~:
    \begin{itemize}
        \rightskip=0pt\leftskip=0pt
        \item \textcolor{blue}{\textbf{Avant l'action}} en nous incitant ou en nous obligeant à la faire ou à la rejeter~;
        \item \textcolor{blue}{\textbf{Pendant l'action}} pour la redresser si elle prend une mauvaise direction, si elle s'éloigne du Bien Commun~;
        \item \textcolor{blue}{\textbf{Après l'action}} pour juger de sa conformité avec le Bien Commun. Ce jugement, loin d'être une mauvaise chose, nous permet alors d'améliorer nos futures actions.
    \end{itemize}
\end{frame}
\begin{frame}{Science sans conscience n'est que ruine de l'âme}
    \begin{block}{Rabelais, extrait de Gargantua et Pantagruel, livre II~:}
    \rightskip=0pt\leftskip=0pt
    «~Mais parce que, selon le sage Salomon, sapience n'entre point en âme malivole\footnote[frame]{Malveillante.} et \textcolor{blue}{science sans conscience n'est que ruine de l'âme}, il te convient servir, aimer et craindre Dieu et en lui mettre toutes tes pensées et tout ton espoir et par foi, formée de charité, être à lui adjoint, en sorte que jamais n'en sois désemparé\footnote[frame]{Séparé.} par péché. Aie suspects les abus du monde. Ne mets ton cœur à vanité, car cette vie est transitoire, mais la parole de Dieu demeure éternellement. Sois serviable à tous tes prochains et les aime comme toi-même. Révère tes précepteurs, fuis les compagnies de gens esquels tu ne veux point ressembler, et, les grâces que Dieu t'a données, icelles ne reçois en vain.~»
\end{block}
\end{frame}
\begin{frame}{La conscience comme lieu de l'alliance du vrai et du bien}
    {
        \begin{figure}
          \centering
          \includegraphics[width = 1\textwidth]{alliance}
        \end{figure}
          }
\end{frame}
\begin{frame}{La conscience comme lieu de l'alliance du vrai et du bien}
    \begin{block}{Thomas d'Aquin, Somme, Iq79,a11,ad2~:}
        \rightskip=0pt\leftskip=0pt
        «~Le vrai et le bien s'implique mutuellement. Car le vrai est un bien, sans quoi il ne serait pas désirable~; et le bien est un vrai, autrement il ne serait pas intelligible. De même donc que l'objet de l'appétit peut être un vrai, considéré comme bien, -- par exemple, lorsqu'on désire connaître la vérité, -- de même l'objet de l'intellect pratique est le bien qui peut être ordonné à l'action, considéré comme vrai. L'intellect pratique en effet connaît la vérité, comme l'intellect spéculatif mais il ordonne à l'action cette vérité connue.~»
    \end{block}
    \begin{itemize}
        \rightskip=0pt\leftskip=0pt
        \item La conscience est donc l'acte qui réalise en moi l'alliance du vrai et du bien~;
        \item D'où l'importance du développement de l'intelligence pour faire le bien~;
        \item Mais aussi du développement de la charité pour mieux connaître.
        \item En suivant ma conscience, je réalise alors dans le monde l'alliance du vrai et du bien.
    \end{itemize}
\end{frame}
\section{La syndérèse}
\begin{frame}{La syndérèse}
    {
        \begin{figure}
          \centering
          \includegraphics[width = 0.95\textwidth]{etincelle}
        \end{figure}
          }    
\end{frame}
\begin{frame}{La syndérèse}
    \begin{block}{La syndérèse est la \textbf{sintilla conscientiae}~:}
        \begin{itemize}
            \rightskip=0pt\leftskip=0pt 
            \item L'Étincelle de la conscience~;
            \item La fine pointe de l'âme.
        \end{itemize}       
    \end{block}
    \begin{exampleblock}{le mot syndérèse vient du grec syntêrêsis}
        \begin{itemize}
            \rightskip=0pt\leftskip=0pt
            \item Il signifie~: \textbf{conserver, garder avec soin}~;
            \item Cicéron traduira ce terme grec en latin par \textbf{conservatio}, l'action de conserver~;
            \item On suppose que l'utilisation de ce terme en philosophie viendrait des stoïciens, même s'il a surtout été repris par Jérôme de Stribon et Thomas d'Aquin.
            \item Il désigne alors la loi primordiale selon laquelle tout être tend à vivre conformément à sa nature.
        \end{itemize}
    \end{exampleblock}
\end{frame}
\begin{frame}{Thomas d'Aquin et la syndérèse}
   \begin{itemize}
        \rightskip=0pt\leftskip=0pt
        \item C'est un habitus naturel, une «~èxis~» naturelle~;
        \item Il est donné avec la nature, et donc non acquis~;
        \item Tout être humain en est doué.
        \item C'est un \textcolor{blue}{\textbf{habitus qui nous incite au bien et nous détourne du mal}}.
        \item En latin~: «~synderesis dicitur instigare ad bonum, et \textcolor{blue}{\textbf{murmurare}} malo~»~;
        \item Elle se donne à nous sous la forme d'\textcolor{blue}{\textbf{un murmure}}, c'est \textcolor{blue}{\textbf{la petite voix de la conscience}}~!
        \item Si nous ne faisons pas silence en nous, nous risquons de ne pas l'entendre~!
    \end{itemize}
\end{frame}
\begin{frame}{Thomas d'Aquin et la syndérèse}
    \begin{block}{La Somme, Iq79a12 réponse~:}
        \rightskip=0pt\leftskip=0pt
        «~C'est pourquoi l'on dit que la syndérèse incite au bien, et proteste contre le mal, lorsque nous nous mettons, à l'aide des premiers principes pratiques, à la recherche de ce qu'il faut faire, et que nous jugeons ce que nous avons trouvé.~»
    \end{block}
    \begin{itemize}
        \rightskip=0pt\leftskip=0pt
        \item La syndérèse est l'intelligence des premiers principes pratiques, et elle se manifeste en nous incitant à faire le bien et en murmurant contre le mal.
        \item Encore faut-il apprendre à l'écouter, puis à décider de lui obéir~!
    \end{itemize}
\end{frame}
\begin{frame}{Thomas d'Aquin et la syndérèse dans le De Veritate q. XVI}
    {
        \begin{figure}
          \centering
          \includegraphics[width = 1\textwidth]{etincelle2}
        \end{figure}
          }    
\end{frame}
\begin{frame}{Thomas d'Aquin et la syndérèse dans le De Veritate q. XVI}
    \rightskip=0pt\leftskip=0pt
    Il se pose 3 questions concernant la syndérèse~:
    \begin{block}{1 La syndérèse est-elle une puissance ou un habitus~?}
        \begin{itemize}
            \rightskip=0pt\leftskip=0pt
            \item C'est une connaissance de la vérité dans le domaine pratique qui se fait sans recherche~;
            \item Elle est naturellement dans l'homme comme la semence de toute connaissance qui va en découler~;
            \item Elle l'est donc sous forme d'habitus (èxis) pour que l'homme l'ait sous la main (ut in promptu), quand il en a besoin.
        \end{itemize}
    \end{block}
\end{frame}
\begin{frame}{Thomas d'Aquin et la syndérèse dans le De Veritate q. XVI}
    \begin{block}{2 La syndérèse peut-elle se tromper~?}
        \begin{itemize}
            \rightskip=0pt\leftskip=0pt
            \item Elle est le \textcolor{blue}{principe permanent} qui résiste à tout mal et consent à tout bien~;
            \item \textcolor{blue}{Elle ne peut donc pas se tromper}~;
            \item \textcolor{orange}{En revanche, dans l'application des principes donnés par elle, notre conscience peut se tromper}~;
            \item \textcolor{red}{Notre conscience peut se tromper dans l'application des principes à des actions particulières}.
            \item C'est \textcolor{red}{le défaut d'attention} de notre conscience à ce que \textcolor{blue}{murmure} la syndérèse, qui fait que notre conscience peut se tromper.
        \end{itemize}
    \end{block}
\end{frame}
\begin{frame}{Thomas d'Aquin et la syndérèse dans le De Veritate q. XVI}
    \begin{block}{3 La syndérèse peut-elle s'éteindre dans une personne~?}
        \rightskip=0pt\leftskip=0pt
            Le verbe \textbf{éteindre} peut avoir 2 sens~:
        \begin{itemize}
            \rightskip=0pt\leftskip=0pt
            \item Soit dans le sens où la lumière que représente la syndérèse pourrait disparaître~;
            \item Soit dans le sens où cette lumière n'arrive pas à éclairer l'action.
        \end{itemize}
    \end{block}
    \begin{exampleblock}{La lumière de la syndérèse ne peut pas disparaître}
        \begin{itemize}
            \rightskip=0pt\leftskip=0pt
            \item Elle appartient à la nature même de l'âme humaine~;
            \item C'est par elle que l'âme humaine est dite intelligente (plus encore que rationnelle ou raisonnable)~;
            \item Elle ne peut donc pas s'éteindre dans ce sens là.
            \item Ou alors cela voudrait dire que l'homme a perdu sa nature humaine~!
        \end{itemize}
    \end{exampleblock}
\end{frame}
\begin{frame}{Thomas d'Aquin et la syndérèse dans le De Veritate q. XVI}
    \begin{alertblock}{En revanche, elle peut ne plus éclairer l'acte}
        \begin{itemize}
            \rightskip=0pt\leftskip=0pt
            \item Le sujet n'arrive plus à suivre cette lumière~:
            \begin{itemize}
                \rightskip=0pt\leftskip=0pt
                \item Perte du libre arbitre, par exemple à cause de la folie~;
                \item Perte de l'usage de la raison, par exemple à cause d'une lésion au cerveau.
            \end{itemize}
            \item Le sujet met en veilleuse sa syndérèse~:
            \begin{itemize}
                \rightskip=0pt\leftskip=0pt
                \item Sa conscience est absorbée par quelque passion~;
                \item Sa conscience est abaissée par quelque habitus mauvais, c'est-à-dire des vices~;
                \item Sa conscience est emportée par de faux raisonnements, ou de mauvaises persuasions.
            \end{itemize}
        \end{itemize}
    \end{alertblock}
\end{frame}
\section{Présentation rapide du problème}
\begin{frame}{Présentation rapide du problème}
    \begin{itemize}
    \rightskip=0pt\leftskip=0pt
\item La syndérèse existe et devrait nous permettre de faire le bien mais
malheureusement nous constatons que le mal existe~;
        \item La conscience peut donc être «~déposée~», comment expliquer cela~?
\item Le fait de «~déposer~» la conscience entraine un problème au niveau de la
conscience morale. Cependant cela entraine aussi un problème au niveau de la
conscience psychologique~!
    \end{itemize}

\end{frame}
\subsection{Fonctionnement que nous devrions avoir}
\begin{frame}{Fonctionnement que nous devrions avoir}
\rightskip=0pt\leftskip=0pt
    {
  \begin{figure}
    \centering
    \includegraphics[width = 1.0\textwidth]{5a}
  \end{figure}
    }
\end{frame}
\subsection{Fonctionnement que nous constatons}
\begin{frame}{Fonctionnement que nous constatons}
    \rightskip=0pt\leftskip=0pt
    {
  \begin{figure}
    \centering
    \includegraphics[width = 1.0\textwidth]{5b2}
  \end{figure}
    }
\end{frame}
\subsection{Explication de ce qui se passe}
\begin{frame}{Explication de ce qui se passe~: schéma général}
      \rightskip=0pt\leftskip=0pt
    {
  \begin{figure}
    \centering
    \includegraphics[width = 1.0\textwidth]{5c1}
  \end{figure}
    }
\end{frame}
\begin{frame}{Explication de ce qui se passe~: les entraves}
      \rightskip=0pt\leftskip=0pt
    {
  \begin{figure}
    \centering
    \includegraphics[width = 1.0\textwidth]{5c2}
  \end{figure}
    }
\end{frame}
\end{document}
